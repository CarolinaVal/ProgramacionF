\documentclass[12pt]{article}
\usepackage[utf8]{inputenc}
\usepackage{amsmath}
\usepackage{graphics}
\author{Carolina Valenzuela Córdova}
\title{\LaTeX}
\date{}
\begin{document}
\title{Reporte Producto 7: Descripción de actividades}
\maketitle{}






La marea es el cambio periódico del nivel del mar producido principalmente por la fuerza de atracción gravitatoria que ejercen el Sol y la Luna sobre la Tierra. Aunque dicha atracción se ejerce sobre todo el planeta, tanto en su parte sólida como líquida y gaseosa, nos referiremos en este artículo a la atracción de la Luna y el Sol, juntos o por separado, sobre las aguas de los mares y océanos. Sin embargo, hay que indicar que las mareas de la litosfera son prácticamente insignificantes, con respecto a las que ocurren en el mar u océano (que pueden modificar su nivel en varios metros) y, sobre todo, en la atmósfera, donde puede variar en varios km de altura, aunque en este caso, es mucho mayor el aumento del espesor de la atmósfera producido por la fuerza centrífuga del movimiento de rotación en la zona ecuatorial (donde el espesor de la atmósfera es mucho mayor) que la modificación introducida por las mareas en dicha zona ecuatorial.\\

 El fenómeno de las mareas es conocido desde la antigüedad. Parece ser que Piteas fue el primero en señalar la relación entre la amplitud de la marea y las fases de la Luna, así como su periodicidad. Plinio el Viejo en su Naturalis Historia describe correctamente el fenómeno y piensa que la marea está relacionada con la Luna y el Sol. Mucho más tarde, Bacon, Kepler y otros trataron de explicar ese fenómeno, admitiendo la atracción de la Luna y del Sol. Pero fue Isaac Newton en su obra Philosophiae Naturalis Principia Mathematica uien dio la explicación de las mareas aceptada actualmente. Más tarde, Pierre Simon-A continuación se recogen los principales términos empleados en la descripción de las mareas:

Marea alta o pleamar: momento en que el agua del mar alcanza su máxima altura dentro del ciclo de las mareas.
Marea baja o bajamar: momento opuesto, en que el mar alcanza su menor altura.Laplace y otros científicos ampliaron el estudio de las mareas desde un punto de vista dinámico.
 Isaac Newton realizó varios estudios científicos del comportamiento de las mareas y calculó la altura de éstas según la fecha del mes, la estación del año y la latitud. Más tarde, Simon Laplace complementó los estudios de Newton.\\


A continuación se recogen los principales términos empleados en la descripción de las mareas:

*Marea alta o pleamar: momento en que el agua del mar alcanza su máxima altura dentro del ciclo de las mareas.\\
*Marea baja o bajamar: momento opuesto, en que el mar alcanza su menor altura.
El tiempo aproximado entre una pleamar y la bajamar es de 6 horas, completando un ciclo de 24 horas 50 minutos.\\
En la última actividad del curso, se analizó un un conjunto de series de tiempo de un sensor que mide: Fecha (mm/dd/aaaa), tiempo (cada 30min), presión (kPa), temperatura del agua (ºC), nivel del mar (metros) y día del año (DOY=Day of Year: 1-365). Tenemos un archivo con datos, que nos ha proporcionado el Dr. Julio César Rodríguez, del Departamento de Agricultura. Los datos se proporcionan en un archivo en formato de Excel, el cual fue necesario modificar para que el programa producido en Fortran puediera leerlo y graficarlo.
Algunas modificaciones consistieron en cambiar el formato de las horas, adecuar el conteo de los días y acomodar todos los datos en una  matriz para que el programa graficara solamente lo que necesitamos, es decir, todas las mareas mínimas y máximas.
Además, como parte de la actividad, también se pedía calcular distancias entre valles y crestas de las gráficas obtenidas a lo largo de la misma.\\

A continuación se presentan las gráficas obtenidas con el programa producido:

En conclusión se observa que las mareas altas y bajas varían con cada día, debido al movimiento de la luna con respecto a la Tierra. 


\end{document}